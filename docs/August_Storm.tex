\documentclass[12pt,a4paper]{article}
\usepackage[utf8]{inputenc}
\usepackage[francais]{babel}
\usepackage[T1]{fontenc}
\usepackage{amsmath}
\usepackage{amsfonts}
\usepackage{amssymb}
\usepackage{graphicx}
\usepackage[left=2cm,right=2cm,top=2cm,bottom=2cm]{geometry}
\begin{document}
\begin{abstract}
Nous analysons les précipitations du 16 août 2022 à Paris quand un orage de grande ampleur s'est abattu sur la ville. Nous utilisons les données d'un pluviomètre à auget basculeur modifié. La modification consiste à enregistrer les instants de bascule de l'auget et non  à comptabiliser le nombre de bascule durant une durée (la plus fréquente étant de 1 heure). Dans le cas d'un orage extrêmement intense mais de courte durée (moins d'une heure et demi) nous montrons l'intérêt de ce genre de modification peu couteuse.
\end{abstract}

Il est tombé $31.5$ mm de pluie entre 16:46:45 et 18:08:21 UTC sur le toit de la terrasse du 5ème étage le 16 Août 2022.
soit $1.36$ heure ou encore 1 heure 21 minutes et 36 secondes. La précipitation horaire moyenne est donc de  23.16 mm/h sur l'IPG. 
cet orage à été enregistré au travers de 126 basculement de l'auget.
Les deux stations voisine dont les données sont en accès libre, celle du LATMOS et celle situé dans le 6ème à saint Germain donne des valeurs proches: $30.8$mm entre 16h et 19h pour la station de Saint Germain des Prés et $35$mm entre 16h et 19h pour la station du LATMOS.

Si on considère un histogramme de pluie tel qu'il est fournit par n'importe quelle station classique avec un pas de temps horaire

Si on améliore progressivement la précision tout en gardant un pas constant on tombe sur le problème des valeurs nulles de précipitation. 

la meilleure solution avec un auget consiste à enregistrer les temps de bascule. l'intensité est alors donnée par
\begin{equation}
I = H/dt 
\end{equation}
ou $H$ correspond à la hauteur de précipitation d'un auget rempli et $dt$ l'intervalle entre les deux bascules.Dans notre cas $H=0.25$mm. 

La limite est ici imposée en théorie par la valeur de H et en pratique par la précision de l'horloge qui dans la plupart des cas est (pour des raisons de coût) de 1 seconde. La valeur maximale $I_{max}$ est donc donnée par
\begin{equation}
I_max = H/1 = H
\end{equation}
soit $I_{max} = 2.5\cdot 10^{-4}$m/s dans notre cas ou encore dans une unité que tout le monde comprend $900$ mm/h

Dans le cas du 16 Août l'orage était réellement exceptionnel la période la plus courte entre deux bascules de l'auget est de 4 secondes soit une intensité de $225$mm/h ou $6.25\cdot 1{-5}$m/s 


\begin{figure}[hbtp]
\centering
\includegraphics[scale=1]{storm_20220816.pdf}
\caption{}
\end{figure}


\end{document}